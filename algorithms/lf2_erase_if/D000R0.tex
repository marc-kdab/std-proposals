\documentclass[11pt]{article}

\usepackage{xcolor}
\usepackage{fullpage}
\usepackage[colorlinks, allcolors=blue]{hyperref}
\usepackage{listings}
\usepackage{parskip}

\lstdefinelanguage{diff}{
  morecomment=[f][\color{blue}]{@@},           % group identifier
  morecomment=[f][\color{red}]{-},             % deleted lines
  morecomment=[f][\color{green!50!black}]{+},  % added lines
  morecomment=[f][\color{magenta}]{---},       % diff header lines
  morecomment=[f][\color{magenta}]{+++},
}

\lstset{
  basicstyle=\footnotesize\ttfamily,
}

\newcommand{\emailaddress}{marc.mutz@kdab.com}
\newcommand{\email}{\href{mailto:\emailaddress}{\emailaddress}}

\date{}
\title{Improving the Return Value of Erase-Like Algorithms}

\begin{document}

\maketitle\vspace{-2cm}

\begin{flushright}
  \begin{tabular}{ll}
  Document \#:&D0xxxR0\\
  Date:       &\date{2017-05-19}\\
  Project:    &Programming Language C++\\
              &Library Evolution Working Group\\
%              &Library Group\\
  Reply-to:   &\author{Marc Mutz} \textless\email\textgreater
  \end{tabular}
\end{flushright}

\section{Introduction}

We propose to change the return type of \cite{LFv2TS} \texttt{erase()}
and \texttt{erase\_if()} algorithms, as well as the \texttt{remove()},
\texttt{remove\_if()} and \texttt{unique()} members of
\texttt{forward\_list} and \texttt{list} from \texttt{void} to
\texttt{size\_t}, returning the number of elements removed.

This restores consistency with long-established API, such as
\texttt{map/set::erase(key\_type)}.

We show that C++17 compilers do not pessimise existing users that
ignore the return value.

\section{Motivation and Scope}

\subsection{[[nodiscard]] Useful Information}

Alexander Stepanov, in his A9 courses\cite{A9}, teaches us not to
throw away useful information, but instead return it from the
algorithm.

With that in mind, look at the following example:
\begin{lstlisting}[language=C++]
std::forward_list<std::shared_ptr<T>> fl = ...;
erase(fl, nullptr);
\end{lstlisting}
Did \texttt{erase()} erase anything? We don't know. The only way we
\emph{can} learn whether the algorithm removed something is to check
the size of the list before and after the algorithm run. For most
containers, that is a valid option, and fast. All \texttt{size()}
methods of STL containers are $O(1)$ these days.

But \texttt{std::forward\_list} has no \texttt{size()}\ldots

We therefore propose to make the algorithms return the number of
removed elements. While it is only really necessary for
\texttt{forward\_list}, we believe that consistency here is more
important than minimalism.

Returning the number of elements also enables convenient one-line
checks:
\begin{lstlisting}[language=C++]
if (erase(lf, nullptr)) {
    // erased some
}
\end{lstlisting}

\subsection{Consistency}

We note that the associative containers have returned the number of
erased elements from their \texttt{erase(key\_type)} member functions
since at least \cite{STL}. This proposal therefore also restores
lost consistency with existing practice.

\section{Impact on the Standard}

Minimal. We propose to change the return value of library functions
from \texttt{void} to \texttt{size\_t}. Existing users expecting no
return value can continue to ignore it.

\section{Proposed Wording}

\subsection{Changes to \cite{cpp17}}

In section \textbf{[forwardlist.overview]}:

\begin{itemize}
\item in paragraph 3, change the \texttt{remove()},
  \texttt{remove\_if()} and \texttt{unique()} return types from
  \texttt{void} to \texttt{size\_t} (four instances).
\end{itemize}

In section \textbf{[forwardlist.ops]}:

\begin{itemize}
\item after paragraphs 11 and 15, change the \texttt{remove()},
  \texttt{remove\_if()} and \texttt{unique()} return types from
  \texttt{void} to \texttt{size\_t} (four instances).
\item after paragraphs 12 and 16, add new paragraph each:
  \begin{quotation}
    \textit{Returns:} The number of elements erased.
  \end{quotation}
\end{itemize}

In section \textbf{[list.overview]}:

\begin{itemize}
\item in paragraph 2, change the \texttt{remove()},
  \texttt{remove\_if()} and \texttt{unique()} return types from
  \texttt{void} to \texttt{size\_t} (four instances).
\end{itemize}

In section \textbf{[list.ops]}:

\begin{itemize}
\item after paragraphs 14 and 18, change the \texttt{remove()},
  \texttt{remove\_if()} and \texttt{unique()} return types from
  \texttt{void} to \texttt{size\_t} (four instances).
\item after paragraphs 15 and 19, add new paragraph each:
  \begin{quotation}
    \textit{Returns:} The number of elements erased.
  \end{quotation}
\end{itemize}

\subsection{Changes to \cite{LFv2TS}}

In section \textbf{[container.erasure.erase\_if]}:

\begin{itemize}
\item replace all \texttt{void} return types with \texttt{size\_t}
\item change paragraph 2 to
  \begin{quotation}
    \textit{Effects:} Equivalent to:
    \begin{lstlisting}[language=C++]
      auto it = remove(c.begin(), c.end(), value);
      auto res = size_t(distance(it, c.end()));
      c.erase(it, c.end());
      return res;
    \end{lstlisting}
  \end{quotation}
\item add new paragraph after each of paragraphs 2, 4, and 6:
  \begin{quotation}
    \textit{Returns:} The number of elements erased.
  \end{quotation}
\item in paragraph 4, insert \texttt{return} between ``Equivalent
  to:'' and ``\texttt{c.remove\_if(}\ldots''.
\item change paragraph 4 to
  \begin{quotation}
    \textit{Effects:} Equivalent to:
    \begin{lstlisting}[language=diff]
+ size_t res = 0;
  for (auto i = c.begin(), last = c.end(); i != last; ) {
    if (pred(*i)) {
      i = c.erase(i);
+     ++res;
    } else {
      ++i;
    }
  }
+ return res;
    \end{lstlisting}
  \end{quotation}
\end{itemize}

In section \textbf{[container.erasure.erase]}:

\begin{itemize}
\item replace all \texttt{void} return types with \texttt{size\_t}
\item change paragraph 2 to
  \begin{quotation}
    \textit{Effects:} Equivalent to:
    \begin{lstlisting}[language=C++]
      auto it = remove(c.begin(), c.end(), value);
      auto res = size_t(distance(it, c.end()));
      c.erase(it, c.end());
      return res;
    \end{lstlisting}
  \end{quotation}
\item add new paragraph after each of paragraphs 2 and 4:
  \begin{quotation}
    \textit{Returns:} The number of elements erased.
  \end{quotation}
\item in paragraph 4, insert \texttt{return} between ``Equivalent to:'' and ``\texttt{erase\_if(}\ldots''.
\end{itemize}

\section{Design Decisions}

\subsection{Open Questions}

Should we return \texttt{Container::size\_type} or
\texttt{std::size\_t} from these functions? We have chosen
\texttt{size\_t} for now, because of brevity, but are fine with
\texttt{size\_type}, too, should the committee favour that.

\subsection{Performance Considerations}

Early reviewers of this proposal expressed concerns that the
calculation of the return value might pessimise the algorithm over the
version that returns \texttt{void}. Tests run on \url{godbolt.org}
show, however, that the assembler instructions generated for the
functions \texttt{counting()} and \texttt{noncounting()} in the
following test were identical for GCC:

\lstinputlisting[language=C++]{code/comparison.cpp}

Clang sometimes formats the code a little differently (same
instructions, grouped differently), without a clear indication which
of the two is better. In Table~\ref{tab:asm}, this is called
\emph{equivalent}.

\begin{table}
  \centering
  \begin{tabular}[t]{|l||c|c|c|}
    \hline
    Container      & GCC 7.1   & Clang 4.0  & MSVC 2017 \\ \hline\hline
    vector         & identical & identical  & --- \\
    deque          & identical & identical  & --- \\
    list           & identical & equivalent & --- \\
    set            & identical & equivalent & --- \\
    unordered\_set & identical & identical  & --- \\
    map            & identical & equivalent & --- \\
    unordered\_map & identical & identical  & --- \\ \hline
  \end{tabular}
  \caption{Assembler Comparison @ \texttt{-O2} (MSVC does not support constexpr-if)}
  \label{tab:asm}
\end{table}

We think it is safe to say that the introduction of the return type
does not pessimise callers that don't need it.

\section{References}
\renewcommand{\section}[2]{}%
\begin{thebibliography}{9}
\bibitem[A9]{A9}
  Alexander Stepanov \emph{et al.}\newline
  \emph{Four Algorithmic Journeys / Efficient Programming With Components /
    Programming Conversations}\newline
  \url{https://www.youtube.com/user/A9Videos/playlists?view=1}

\bibitem[SGI STL]{STL}
  Alexander Stepanov \emph{et al.}\newline
  \emph{Associative Container}\newline
  in: \emph{Standard Template Library Programmer's Guide}\newline
  \url{https://www.sgi.com/tech/stl/AssociativeContainer.html}

\bibitem[N4600]{LFv2TS}
  Geoffrey Romer (editor)\newline
  \emph{Working Draft, C++ Extensions for Library Fundamentals, Version 2}\newline
  \url{http://open-std.org/JTC1/SC22/WG21/docs/papers/2016/n4600.html}

\bibitem[N4659]{cpp17}
  Richard Smith (editor)\newline
  \emph{Working Draft, Standard for Programming Language C++}\newline
  \url{http://www.open-std.org/jtc1/sc22/wg21/docs/papers/2017/n4659.pdf}
\end{thebibliography}

\end{document}

