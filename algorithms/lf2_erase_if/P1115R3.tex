\documentclass[11pt]{article}

\usepackage{xcolor}
\usepackage{fullpage}
\usepackage[colorlinks, allcolors=blue]{hyperref}
\usepackage{listings}
\usepackage{parskip}

\lstdefinelanguage{diff}{
  morecomment=[f][\color{blue}]{@@},           % group identifier
  morecomment=[f][\color{red}]{-},             % deleted lines
  morecomment=[f][\color{green!50!black}]{+},  % added lines
  morecomment=[f][\color{magenta}]{---},       % diff header lines
  morecomment=[f][\color{magenta}]{+++},
}

\lstset{
  basicstyle=\footnotesize\ttfamily,
}

\setcounter{section}{-1}

\newcommand{\emailaddress}{marc.mutz@kdab.com}
\newcommand{\email}{\href{mailto:\emailaddress}{\emailaddress}}
\newcommand{\wgpaper}[1]{\href{https://wg21.link/#1}{#1}}
\newcommand{\isref}[1]{\textbf{[\wgpaper{#1}]}}
\newcommand{\cst}{\texttt{\textit{<container>}::size\_type}}

\date{}
\title{Improving the Return Value of Erase-Like Algorithms II:\\ Free
  \texttt{erase}/\texttt{erase\_if}}

\begin{document}

\maketitle\vspace{-2cm}

\begin{tabular}{ll}
% Document \#:&P0646R0\\
  Document \#:&R1115R3\\
  Date:       &\today\\
  Project:    &Programming Language C++\\
              &Library Working Group\\
%              &Library Group\\
  Reply-to:   &\author{Marc Mutz} \textless\email\textgreater
\end{tabular}
\vspace{1cm}
\begin{abstract}
  We propose to change the return type of \texttt{erase()} and
  \texttt{erase\_if()} free functions from \texttt{void} to \cst,
  returning the number of elements removed. This restores consistency
  with long-established API, such as
  \texttt{map/set::erase(key\_type)}, as well as the recent changes to
  \texttt{forward\_}/\texttt{list::remove()}.
\end{abstract}


\tableofcontents

\section{Change History}

This is a spin-off and revision of \wgpaper{P0646R0} at the request of
LWG in Rapperswil to work around the problem of LFv3 not having opened
shop in Rapperswil, yet.

\subsection{Changes from P1115R2}

\begin{enumerate}
\item Fixed \texttt{erase()} wording (was missing \texttt{value}).
\item Fixed an IS section reference in Section~\ref{wording}.
\item Rebased onto \cite{N4835}.
\end{enumerate}

\subsection{Changes from P1115R1}

\begin{enumerate}
\item Fixed \texttt{erase()} wording (contained \texttt{pred}, but shouldn't).
\item Changed associative containers' algorithm from incrementing a running count to subtracting original and new sizes.
\end{enumerate}

\subsection{Changes from P1115R0}

\begin{enumerate}
\item Fixed an IS section reference in Section~\ref{wording}.
\item Rebased onto \wgpaper{N4830}.
\end{enumerate}

\subsection{Changes from P0646R0}

\begin{enumerate}
\item Removed changes to the IS draft, as these continued as
  \wgpaper{P0646R1} (which has since been adopted in Rapperswil).
\item Changed the return type from \texttt{size\_t} to
  \cst, as requestd by LEWG in Toronto.
\item Rebased on IS draft, as the target of this proposal has since
  been merged into it from the LFv2 TS.
\item Added feature test macro.
\end{enumerate}

\section{Motivation and Scope}

This section is copied from \wgpaper{P0646R1}, so readers familiar
with that paper can skip these paragraphs.

\subsection{[[nodiscard]] Useful Information}

Alexander Stepanov, in his A9 courses\cite{A9}, teaches us not to
throw away useful information, but instead return it from the
algorithm.

With that in mind, look at the following example:
\begin{lstlisting}[language=C++]
std::forward_list<std::shared_ptr<T>> fl = ...;
erase(fl, nullptr);
\end{lstlisting}
Did \texttt{erase()} erase anything? We don't know. The only way we
\emph{can} learn whether the algorithm removed something is to check
the size of the list before and after the algorithm run. For most
containers, that is a valid option, and fast. All \texttt{size()}
methods of STL containers are $O(1)$ these days.

But \texttt{std::forward\_list} has no \texttt{size()}\ldots

We therefore propose to make the algorithms return the number of
removed elements. While it is only really necessary for
\texttt{forward\_list}, we believe that consistency here is more
important than minimalism.

Returning the number of elements also enables convenient one-line
checks:
\begin{lstlisting}[language=C++]
if (erase(fl, nullptr)) {
    // erased some
}
\end{lstlisting}

\subsection{Consistency}

In Rapperswil, the committee accepted \wgpaper{P0646R1}, which changed
the \texttt{list} and \texttt{forward\_list} member algorithms
\texttt{remove}/\texttt{\_if} and \texttt{unique} to return the number
of elements erased. This paper applies the same logic to the
non-member versions of these algorithms.

We note that the associative containers have returned the number of
erased elements from their \texttt{erase(key\_type)} member functions
since at least \cite{STL}. This proposal therefore also restores
lost consistency with existing practice.

\section{Impact on the Standard}

Minimal. We propose to change the return value of library functions
from \texttt{void} to \texttt{size\_type}. Existing users of the LFv2
versions expecting no return value can continue to ignore it. In
particular, this is one of the changes explicitly mentioned in
\cite{P0921R2}.

Strictly speaking, the change is source-incompatible: Existing code
which assumes that the algorithms return \texttt{void} might fail to
compile. This can e.g.\ come up in situations where the C++ user
explicitly specialized these algorithms. However, all such code will
so far have used the LFv2 versions of these algorithms, which are in a
different namespace.

For the same reason, there is no binary-compatibility issue here: the
algorithms in LFv2 were specified in namespace
\texttt{std::experimental}, while the changed algorithms will be in
\texttt{std} directly.

\section{Proposed Wording}
\label{wording}

The following changes are relative to \cite{N4835}:

\begin{itemize}
\item In \isref{version.syn}, adjust the value of the
  "\texttt{\_\_cpp\_lib\_erase\_if}" macro to match the date of
  application of this paper to the IS draft.

\item In each of \isref{string.syn},
  \isref{string.erasure},\\ \isref{deque.syn},
  \isref{forward.list.syn}, \isref{list.syn}, \isref{vector.syn},
  \\\isref{deque.erasure}, \isref{forward.list.erasure},
  \isref{list.erasure}, \isref{vector.erasure},\\
  \isref{associative.map.syn}, \isref{associative.set.syn},
  \isref{unord.map.syn}, \isref{unord.set.syn},\\\isref{map.erasure},
  \isref{multimap.erasure}, \isref{set.erasure},
  \isref{multiset.erasure},\\ \isref{unord.map.erasure},
  \isref{unord.multimap.erasure},\\ \isref{unord.set.erasure},
  \isref{unord.multiset.erasure}:
  
For each \texttt{erase(\textit{<container>}\& c, \ldots)} and
\texttt{erase\_if(\textit{<container>}\& c, \ldots)} function, change
the return type from \texttt{void} to \texttt{typename \cst}.

\item In each of \isref{string.erasure}, \isref{deque.erasure},
  \isref{vector.erasure}, change paragraphs 1 as follows:
\begin{lstlisting}[language=diff]
- Effects: Equivalent to: c.erase(remove(c.begin(), c.end(), value), c.end());
+ Effects: Equivalent to:
+    auto it = remove(c.begin(), c.end(), value);
+    auto r = distance(it, c.end());
+    c.erase(it, c.end());
+    return r;
\end{lstlisting}

\item In each of \isref{string.erasure}, \isref{deque.erasure},
  \isref{vector.erasure}, change paragraphs 2 as follows:
\begin{lstlisting}[language=diff]
- Effects: Equivalent to: c.erase(remove_if(c.begin(), c.end(), pred), c.end());
+ Effects: Equivalent to:
+    auto it = remove_if(c.begin(), c.end(), pred);
+    auto r = distance(it, c.end());
+    c.erase(it, c.end());
+    return r;
\end{lstlisting}

\item In each of \isref{forward.list.erasure}, \isref{list.erasure},
  in paragraphs 1 and 2, add ``\texttt{return} '' between ``Equivalent
  to:'' and the start of the code.

\item In each of  \isref{map.erasure},
  \isref{multimap.erasure}, \isref{set.erasure},
  \isref{multiset.erasure},\\ \isref{unord.map.erasure},
  \isref{unord.multi\-map.erasure}, \isref{unord.set.erasure},\\
  \isref{unord.multiset.erasure}:
  
  Change paragraphs 1 as indicated:

\begin{lstlisting}[language=diff]
+ auto original_size = c.size();
  for (auto i = c.begin(), last = c.end(); i != last; ) {
    if (pred(*i)) {
      i = c.erase(i);
    } else {
      ++i;
    }
  }
+ return original_size - c.size();
\end{lstlisting}

\end{itemize}

\subsection{Feature Test Macro}

No new macro is necessary.

\section{Design Decisions}

\subsection{\texttt{size\_t} vs.\ \texttt{size\_type}}

Should we return \cst{} or \texttt{std::size\_t} from these functions?
\wgpaper{P0646R0} chose \texttt{size\_t}, for brevity, but LEWG in
Toronto favoured \texttt{size\_type}, so this is what's proposed now.

\subsection{Performance Considerations}

Please refer to \wgpaper{P0646R0} for a detailed analysis. TL;DR: We
believe that returning the number of elements removed does not pessimise
callers that don't need it.

\section{Acknowledgements}

We thank the reviewers of draft versions of the original proposal and
the participants of the associated discussion on
\url{std-proposals@isocpp.org} and LWG in Rapperswil for their input:
Sean Parent, Arthur O'Dwyer, Nicol Bolas, Ville Voutilainen, Casey
Carter, Milian Wolff, Andr\'e Somers, Jonathan Wakely,
Walter~E.~Brown. All remaining errors are ours.

%\section{References}
%\renewcommand{\section}[2]{}%
\begin{thebibliography}{9}
\bibitem[A9]{A9}
  Alexander Stepanov \emph{et al.}\newline
  \emph{Four Algorithmic Journeys / Efficient Programming With Components /
    Programming Conversations}\newline
  \url{https://www.youtube.com/user/A9Videos/playlists?view=1}

\bibitem[SGI STL]{STL}
  Alexander Stepanov \emph{et al.}\newline
  \emph{Associative Container}\newline
  in: \emph{Standard Template Library Programmer's Guide}\newline
  \url{https://www.sgi.com/tech/stl/AssociativeContainer.html} (accessed 2017-06-01)

\bibitem[N4835]{N4835}
  Richard Smith (editor)\newline
  \emph{Working Draft: Standard for Programming Language C++}\newline
  \url{http://wg21.link/N4835}

\bibitem[P0921R2]{P0921R2} Titus Winters\newline
  \emph{Standard Library Compatibility}\newline
  \url{http://www.open-std.org/jtc1/sc22/wg21/docs/papers/2018/p0921r2.pdf}
\end{thebibliography}

\end{document}

