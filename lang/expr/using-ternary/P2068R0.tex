\documentclass[11pt]{article}

\usepackage{xcolor}
\usepackage{fullpage}
\usepackage[colorlinks, allcolors=blue]{hyperref}
\usepackage{listings}
\usepackage{parskip}
\usepackage{amssymb}

\lstdefinelanguage{diff}{
  morecomment=[f][\color{blue}]{@@},           % group identifier
  morecomment=[f][\color{red}]{-},             % deleted lines
  morecomment=[f][\color{green!50!black}]{+},  % added lines
  morecomment=[f][\color{magenta}]{---},       % diff header lines
  morecomment=[f][\color{magenta}]{+++},
}

\lstset{
  basicstyle=\footnotesize\ttfamily,
}

\newcommand{\emailaddress}{marc.mutz@kdab.com}
\newcommand{\email}{\href{mailto:\emailaddress}{\emailaddress}}

\newcommand{\wgpaper}[1]{\href{https://wg21.link/#1}{#1}}
\newcommand{\isref}[1]{\textbf{[\wgpaper{#1}]}}
\newcommand{\isnref}[2]{\textbf{[\href{https://wg21.link/#1\##2}{#1}]/#2}}

\date{}
\title{Using \texttt{?:} to reduce the scope of constexpr-if}

\begin{document}

\maketitle\vspace{-2cm}

\begin{tabular}{ll}
  Document \#:&P2068R0\\
  Date:       &\today\\
  Project:    &Programming Language C++\\
              &EWG Incubator\\
%              &Library Group\\
  Reply-to:   &\author{Marc Mutz} \textless\email\textgreater
\end{tabular}
\vspace{1cm}
\begin{abstract}
  D's \emph{static~if}, unlike C++17's \emph{constexpr~if}, does not
  introduce scoping. Andrei Alexandrescu has repeatedly highlighted
  this as a major enabler of \emph{static~if} over \emph{constexpr~if}.

  In many cases, this feature of \emph{static~if} is used to
  conditionally select a type. Since the \emph{static~if} scoping
  rules are very alien to C++, we propose to allow the conditional
  operator (which would be implicitly \texttt{constexpr}) on the
  right-hand-side of a \emph{using-declaration}.

\begin{lstlisting}[language=c++]
  template <bool B, typename T, typename F>
  using conditional_t = B ? T : F ;

  template <bool B, typename T, typename F>
  struct conditional { using type = conditional_t<B,T,F>; };
\end{lstlisting}

This greatly reduces the need to revert to template argument pattern
matching (or library wrappers around it) to use conditionals in
template meta programming, and therefore the need for a
\emph{static~if} with D's semantics.
\end{abstract}


%\tableofcontents

\section{Motivation and Scope}

\subsection{Efficient Type Selection}

Vittorio Romeo started his 2016 CppCon talk\cite{VR16} with the following example from D:

\begin{lstlisting}[language=C++]
  template INT(int i) {
      static if (i == 32)
          alias INT = int;
      static if (i == 16)
          alias INT = short;
      else
          static assert(0);
  }
\end{lstlisting}

The best we can do in C++20 is

\begin{lstlisting}[language=C++]
  template <int i>
  using INT = std::conditional_t<i == 32, int,
              std::conditional_t<i == 16, short, std::experimental::nonesuch>;
\end{lstlisting}

This proposal suggests to allow the following instead:

\begin{lstlisting}[language=C++]
  template <int i>
  using INT = i == 32 ? int :
              i == 16 ? short :
              /*else*/  static_assert(dependent_false_v<i>, "no such type") ;
\end{lstlisting}

Andrei Alexandrescu showed the following code in this 2018 Meeting C++
Keynote\cite{AA18}, slightly edited for brevity:

\begin{lstlisting}[language=c++]
  template <class K, class V, size_t maxLength>
  struct RobinHashTable {
    static if (maxLength < 0xFFFE) {
      using CellIdx = uint16_t;
    } else {
      using CellIdx = uint32_t;
    }

    static if (sizeof(K) % 8 < 7) {
      struct KV {
        K k;
        uint8_t cellData;
        V v;
      };
    } else {
      struct KV {
        K k;
        V v;
        uint8_t cellData;
      };
    }
  };
\end{lstlisting}

In C++20, one would have to define both \texttt{struct KV1} and
\texttt{struct KV2} and then alias \texttt{KV} to one of them, using
\texttt{std::conditional\_t}. Instead of this, we simply present what
this proposal suggests to allow:

\begin{lstlisting}[language=c++]
  template <class K, class V, size_t maxLength>
  struct RobinHashTable {
    using CellIdx = maxLength < 0xFFFE ? uint16_t : uint32_t;
    using KV = sizeof(K) % 8 < 7) ? struct { K k; uint8_t cellData; V v; } :
               /* else */           struct { K k; V v; uint8_t cellData; } ;
  };
\end{lstlisting}

The discarded branch would have the same semantics as those of
discarded \emph{constexpr~if} branches.

\section{Impact on the Standard}

Minimal. The syntax we propose to make valid was ill-formed before.

\section{Proposed Wording}

The following is just a quick sketch. More detailed wording can be
provided if the EWG Incubator finds value in this proposal.

It seems that the changes necessary are local to
\emph{using-declarator}. The ``normal'' ternary operator wording in
\isref{expr.cond} is unaffected.

For the first example, we'd need to allow \texttt{static\_assert} in
\emph{declarator-list}.

Something like this:

  \emph{
  using-declarator:\\
  \hspace*{1cm} \texttt{typename$_{opt}$} nested-name-specifier unqualified-id\\
  \hspace*{1cm} \colorbox{green}{static-assert-declaration (mod semicolon)}\\
  \hspace*{1cm} \colorbox{green}{logical-or-expression \texttt? using-declarator \texttt: using-declarator}
  }

where the \emph{logical-or-expression} must meet ``the value of the
condition shall be a contextually converted constant expression of
type \texttt{bool};''

\iffalse
\subsection{Changes to \cite{cpp2a}}

In \isnref{class.base.init}1, as well as in \isref{gram.class}, change
the production for \emph{ctor-initalizer} as indicated:

  \emph{
  ctor-initalizer:\\
  \hspace*{1cm}: mem-initializer-list\colorbox{green}{ ,$_{opt}$}
  }
\fi

\subsection{Feature Macro}

We propose to use a new macro,
\texttt{\_\_cpp\_using\_conditonal\_operator}, to indicate an
implementation's support for this feature.

\iffalse
\section{Design Decisions}

\subsection{Extension to other lists}

One of the points that was raised in the \texttt{std-proposals} 2015
discussion was whether to allow trailing commas for all lists,
incl.\ e.g.\ function arguments. We welcome such discussion, but deem
it outside the scope of this proposal for at least its first
iteration, mainly because there is no general ``list'' grammar
production in the standard to which a change could be uniformly
applied, and other types of lists have not seen the same kinds effort
being spent on work-arounds that \emph{ctor-initalizer} has
``enjoyed''.

\section{Acknowledgements}

Arthur O'Dwyer provided the wording on which this proposal is based.\cite{std}
\fi

\section{References}
\renewcommand{\section}[2]{}%
\begin{thebibliography}{9}
\bibitem{VR16}
  Vittorio Romeo\newline
  CppCon 2016: ``Implementing `static` control flow in C++14''\newline
  \url{https://youtu.be/aXSsUqVSe2k?t=128}
\bibitem{AA18}
  Andrei Alexandrescu\newline
  Meeting C++ 2018: ``The next big Thing ``\newline
  \url{https://youtu.be/tcyb1lpEHm0?t=2716}
\bibitem[N4820]{cpp2a}
  Richard Smith (editor)\newline
  \emph{Working Draft: Standard for Programming Language C++}\newline
  \url{http://wg21.link/N4820}
\end{thebibliography}

\end{document}

