\documentclass[11pt]{article}

\usepackage{xcolor}
\usepackage{fullpage}
\usepackage[colorlinks, allcolors=blue]{hyperref}
\usepackage{listings}
\usepackage{parskip}
\usepackage{amssymb}

\lstdefinelanguage{diff}{
  morecomment=[f][\color{blue}]{@@},           % group identifier
  morecomment=[f][\color{red}]{-},             % deleted lines
  morecomment=[f][\color{green!50!black}]{+},  % added lines
  morecomment=[f][\color{magenta}]{---},       % diff header lines
  morecomment=[f][\color{magenta}]{+++},
}

\lstset{
  basicstyle=\footnotesize\ttfamily,
}

\newcommand{\emailaddress}{marc.mutz@kdab.com}
\newcommand{\email}{\href{mailto:\emailaddress}{\emailaddress}}

\newcommand{\wgpaper}[1]{\href{https://wg21.link/#1}{#1}}
\newcommand{\isref}[1]{\textbf{[\wgpaper{#1}]}}
\newcommand{\isnref}[2]{\textbf{[\href{https://wg21.link/#1\##2}{#1}]/#2}}

\date{}
\title{More flexible \texttt{optional::value\_or()}}
\author{Marc Mutz}

\begin{document}

\maketitle%\vspace{-2cm}

\begin{tabular}{ll}
  Document \#:&N3887\\
  Date:       &\today\\
  Project:    &Programming Language C++\\
              &Library Evolution Group Incubator\\
%              &Library Group\\
  Reply-to:   &\author{Marc Mutz} \textless\email\textgreater
\end{tabular}
\vspace{1cm}
\begin{abstract}
  We propose to extend the \texttt{value\_or()} member function
  template in \texttt{optional}, so as to make requesting
  default-constructed values easier:
  \begin{lstlisting}[language=c++]
    // now
    opt.value_or(Type{});
    // proposed:
    opt.value_or({});
    opt.value_or_make();
  \end{lstlisting}
  This brings \texttt{value\_or()} in line with what other functions
  (most prominently \texttt{exchange()}), do.
\end{abstract}

\tableofcontents

%\setcounter{section}{-1}
%\section{Change History}

\section{Motivation and Scope}

When using \texttt{optional::value\_or()}, more often than not, the
fall-back value passed is some form of default-constructed value:

\begin{lstlisting}[language=c++]
  optional<int> oi = ~~~;       // (1)
  use(oi.value_or(0));
  optional<bool> ob = ~~~;      // (2)
  use(ob.value_or(false));
  optional<string> os = ~~~;    // (3)
  use(os.value_or(nullptr));    // (a)
  use(os.value_or(""));         // (b)
  use(os.value_or({}));         // (c)
  use(os.vlue_or(string{});     // (d)
  optional<vector<string>> ov = ~~~;
  use(ov.value_or(~~~???~~~));  // (4)
\end{lstlisting}

While this works fine in case of built-in types (1, 2), it already
fails to be convenient when the payload type is a user-defined type
without literals.

\subsection{How the C++ Developer Became a Gardener}

Here's the tale of a C++ developer trying to use \texttt{value\_or()}
in the \texttt{string} case (3): The developer first tries to use
\texttt{nullptr} (a), which crashes on him at runtime due to
\isnref{char.traits.require}1 in conjunction with
\isnref{string.cons}{13}. The next try (b) succeeds, but may invoke an
unnecessary ``\texttt{strlen}'', so he's told in review to use the
\texttt{string} default constructor instead. So the developer tries
(c) which fails to compile because \texttt{\{\}} fails to deduce the
template argument of \texttt{value\_or()}, which is not defaulted, as
e.g.\ the second argument of \texttt{exchange()} is. Grumpily, the
developer caves in and repeats the type name of the
\texttt{optional}'s \texttt{value\_type} (d).

The next day, he's asked to use a \texttt{optional<vector<string>>}
(4) and decides to quit and become a gardener instead.

We propose two different, orthogonal, solutions to the problem:
\begin{itemize}
\item Default the \texttt{value\_or} template argument, so
  \texttt{value\_or(\{\})} works, and/or
\item Add an emplacement-like function \texttt{value\_or\_make(auto\&\&...)}, so
  that \texttt{value\_or\_make()} works.
\end{itemize}

\subsection{Defaulting \texttt{value\_or()}'s template argument}
\label{sec:defaulting}

With this change, we'd like to ensure that \texttt{value\_or(\{\})}
works, like \texttt{exchange(\textit{var}, \{\})} does.

We can't just default like this:

\begin{lstlisting}[language=c++]
  template <typename T>
  class optional {
  public:
      ~~~~
      template <typename U = T>
      T value_or(U&&) const;
  };
\end{lstlisting}

as that would prevent moving the argument into the return value when
\texttt{T} is cv-qualified (as in \texttt{optional<const string>}). It
follows that we need to remove cv-qualifiers. We don't need to remove
references, as \texttt{optional<T\&>} is ill-formed. If and when
optional references become supported, this needs to be rethought.

\begin{lstlisting}[language=c++]
  template <typename T>
  class optional {
  public:
      ~~~~
      template <typename U = remove_cv_v<T>>
      T value_or(U&&) const;
  };
\end{lstlisting}

This enables developers to write \texttt{value\_or(\{\})}, which is
self-explanatory, as long as you know \texttt{value\_or()} as
currently specified.

\subsection{Adding emplace-like \texttt{value\_or\_make()}}
\label{sec:emplace}

The second change was suggested to the author in very early
discussions on the LEWGI reflector: If \texttt{value\_or()} was a
variadic emplace-like function, then \texttt{opt.value\_or()} would
return a default-constructed value if \texttt{opt} is not engaged.

While this extension would be SC and BC\footnote{The variadic version
  could overload the existing unary version by constraining the
  variadic version to \texttt{sizeof...(Args) != 1}}, this author does
not believe that \texttt{value\_or()} is a good name for such a
function: What does \texttt{opt.value\_or()} \emph{look like}? Can a
developer that knows \texttt{value\_or()} as currently specified make
sense of this expression? This author doubts that very much. To him,
this looks like ``value or nothing''. Then what's the ``nothing''
that's being returned?  Another \texttt{optional} specialisation?

So, it seems to this author that just making \texttt{value\_or()}
emplace-like would be counter-intuitive, but at the same time such
functionality could be useful. E.g., even if \texttt{value\_or(\{\})}
was enabled (as per Section~\ref{sec:defaulting}), that call would
create a default-constructed \texttt{T} which is then moved into the
return value, instead of default-constructing the return value
directly.

Taking a cue from existing factory functions in the standard, this
author ended up with \texttt{value\_or\_make()} as the suggested name
for the variadic function.

\section{Impact on the Standard}

Only positive. Expressions enabled by this proposal make the use of
\texttt{optional::value\_or()} easier and more consistent with the
rest of the standard library. At the same time, no existing code is
broken.

\section{Proposed Wording}

All wording is relative to \cite{cpp2a}:

\begin{itemize}
\item In \isref{version.syn}, add a feature macro
  \texttt{\_\_cpp\_lib\_optional\_value\_or} with the usual value and
  comment ``\texttt{// also in <optional>}''.
\item Change \isref{optional.optional} as indicated:
  \begin{lstlisting}[language=diff]
    constexpr const T&& value() const&&;
-   template<class U> constexpr T value_or(U&&) const&;
-   template<class U> constexpr T value_or(U&&) &&;
+   template<class U=remove_cv_t<T>> constexpr T value_or(U&&) const&;
+   template<class U=remove_cv_t<T>> constexpr T value_or(U&&) &&;
+   template<class... Args> constexpr T value_or_make(Args&&... args) const&;
+   template<class... Args> constexpr T value_or_make(Args&&... args) &&;

    // [optional.mod], modifiers
  \end{lstlisting}
\item Apply the above \texttt{remove\_cv\_t<T>} default argument also
  to the declarations of \texttt{value\_or()} just above
  \isnref{optional.observe}{17} and \isnref{optional.observe}{19}.
\item At the end of \isref{optional.observe}, add:\\
  \begin{tabular}{lll}
    \\
    \multicolumn3l{\texttt{template<class...\ Args> constexpr T value\_or\_make(Args\&\&...\ args) const\&}}\\\\
    &\multicolumn2l{\textit{Mandates:} \texttt{is\_copy\_constructible\_v<T> \&\& is\_constructible\_v<T, Args...>} is \texttt{true}.}\\
    &\multicolumn2l{\textit{Effects:} Equivalent to:}\\
    &&\texttt{return bool(*this) ?\ **this :\ T(std::forward<Args>(args)...);}\\
    \\
    \multicolumn3l{\texttt{template<class...\ Args> constexpr T value\_or\_make(Args\&\&...\ args) \&\&}}\\\\
    &\multicolumn2l{\textit{Mandates:} \texttt{is\_move\_constructible\_v<T> \&\& is\_constructible\_v<T, Args...>} is \texttt{true}.}\\
    &\multicolumn2l{\textit{Effects:} Equivalent to:}\\
    &&\texttt{return bool(*this) ?\ std::move(**this) :\ T(std::forward<Args>(args)...);}\\
  \end{tabular}
\end{itemize}

\section{Design Decisions}

If all we wanted was to make it easier to return a default-constructed
T, we could just add a new function
\texttt{value\_or\_default\_initialized()}. This is not proposed,
because it does not address the consistency concern with
\texttt{exchange()}.

As mentioned in Section~\ref{sec:emplace}, just making
\texttt{value\_or()} variadic leaves a lot to be desired: while
\texttt{opt.value\_or(0xff, 0xff, 0xff)} works reasonably well for a
\texttt{optional<color>}, it doesn't really work for default
construction, which is the driver behind this proposal. So this author
does not propose to make \texttt{value\_or()} variadic, but suggests
to choose a different name for this functionality.

\section{Acknowledgements}

The author would like to thank all participants of the LEWG(I)
reflector discussion that led to this proposal, esp.\ Andrzej
Krzemienski for confirming that \texttt{value\_or()}'s non-defaulted
template parameter was not a conscious omission.

\section{References}
\renewcommand{\section}[2]{}%
\begin{thebibliography}{9}
\bibitem[N4861]{cpp2a}
  Richard Smith (editor)\newline
  \emph{Working Draft: Standard for Programming Language C++}\newline
  \url{http://wg21.link/N4861}
\end{thebibliography}

\end{document}

