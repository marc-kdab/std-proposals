\documentclass[11pt]{article}

\usepackage{xcolor}
\usepackage{fullpage}
\usepackage[colorlinks, allcolors=blue]{hyperref}
\usepackage{listings}
\usepackage{parskip}
\usepackage{amssymb}

\lstdefinelanguage{diff}{
  morecomment=[f][\color{blue}]{@@},           % group identifier
  morecomment=[f][\color{red}]{-},             % deleted lines
  morecomment=[f][\color{green!50!black}]{+},  % added lines
  morecomment=[f][\color{magenta}]{---},       % diff header lines
  morecomment=[f][\color{magenta}]{+++},
}

\lstset{
  basicstyle=\footnotesize\ttfamily,
}

\newcommand{\emailaddress}{marc.mutz@kdab.com}
\newcommand{\email}{\href{mailto:\emailaddress}{\emailaddress}}

\newcommand{\wgpaper}[1]{\href{https://wg21.link/#1}{#1}}
\newcommand{\isref}[1]{\textbf{[\wgpaper{#1}]}}
\newcommand{\isnref}[2]{\textbf{[\href{https://wg21.link/#1\##2}{#1}]/#2}}

\newcommand{\lockx}{\texttt{lock\_exclusive()}}
\newcommand{\uptr}{\texttt{unique\_ptr}}
\newcommand{\wptr}{\texttt{weak\_ptr}}
\newcommand{\sptr}{\texttt{shared\_ptr}}
\newcommand{\sptrlockx}{\texttt{\sptr::\lockx}}
\newcommand{\sptrcT}{\texttt{\sptr<const T>}}
\newcommand{\makeshared}{\texttt{make\_shared}}

\date{}
\title{Re-Gaining Exclusive Ownership from \sptr{}s}

\begin{document}

\maketitle\vspace{-2cm}

\begin{tabular}{ll}
  Document \#:&P1116R0\\
  Date:       &\today\\
  Project:    &Programming Language C++\\
              &SG1 (for implementability)\\
              &Library Evolution Working Group (for API review)\\
%              &Library Group\\
  Reply-to:   &\author{Marc Mutz} \textless\email\textgreater
\end{tabular}
\vspace{1cm}
\begin{abstract}
  We propose to add a function, \lockx, to \sptr{} to re-gain
  exclusive ownership of the payload. Exclusive ownership is defined
  as a sole \sptr{} owning, combined with no \wptr{}s referencing, the
  resource. The reason to exclude \wptr{}s is that the presence of a
  \wptr{} (e.g.\ in a separate thread of execution) could materialize
  new \sptr{}s at any time, which would invalidate the exclusive
  ownership state.

  The primary use-case is efficient copy-on-write implementations
  using \sptrcT{} to hide the details of ref-counting, a la Sean
  Parent's \texttt{document} example from \cite{Seasoning}. As long as
  the objects involved are immutable, mutation is performed by copying
  from the existing state and modifying the state before storing in
  the new \sptrcT{}. The existing \texttt{unique()} function is not
  sufficient for this purpose, because it does not take the existence
  of associated \wptr{}s into account, which could materialize new
  \sptr{}s at any time from a different thread.
\end{abstract}


\tableofcontents

\section{Motivation and Scope}

\subsection{Efficient Copy-on-Write}

Sean Parent, in \cite{Seasoning}, introduced a pattern of using
\sptrcT{} to share common data between different versions
of a document, without the need to deep copy. In this way, he can
implement undo as a simple \texttt{vector<document>} even with very
large \texttt{document}s.

\sptrcT{} works well for this as long as one works with immutable
objects. But it is well-known that actions\footnote{modifying an
  object's state in-place} can often be more efficient than
transformations\footnote{copying an object's state, making changes,
  and then creating a new object with the new state, leaving the old
  object alone}, so immutability of objects may not always be desired.

Let's look at a typical mutator of a class that uses \sptrcT{} to store
data:

\begin{lstlisting}[language=C++,caption={Typical mutator of a class that uses \sptrcT}]
  shared_ptr<const Data> m_data;

  void set_foo(T foo) {
    auto uniq = clone_or_new(m_data);
    uniq->foo = std::move(foo);
    m_data = std::move(uniq);
  }
\end{lstlisting}

where \texttt{clone\_or\_new} might be implemented like this:

\begin{lstlisting}[language=C++,caption={Inefficient implementation of \texttt{clone\_or\_new()}}]
  static auto clone_or_new(const shared_ptr<const Data> & sp) {
    return sp ? make_shared<Data>(*sp) : make_shared<Data>();
  }
\end{lstlisting}

This is the classical copy-on-write implementation, with a drawback:
every setter allocates, even if it just sets an \texttt{int}
field. This is clearly not acceptable in a production-quality
implementation, in particular for APIs that employ something like Qt's
\cite{qt} Property-Based Design \cite{property_based}, which calls for
preferring setters over long lists of contructor arguments:

\begin{lstlisting}[language=C++,caption={Comparison between Qt~3 and Qt~4 slider creation}]
  // Qt 3:
  QSlider qt3(0, 100, 50, 10); // allocs once, but unreadable
  
  // Qt 4+:
  QSlider slider;          // allocates, because all properties
                           // have defined defaults
  slider.setRange(0, 100); // would allocate
  slider.setPageStep(10);  // would allocate
  slider.setValue(50);     // would allocate
\end{lstlisting}

The obvious optimisation is to not create a new copy of the data if
we're the sole owner:

\begin{lstlisting}[language=C++,label={lst:clone-or-new-using-unique},caption={\texttt{clone\_or\_new()} using \texttt{unique()}}]
  static auto clone_or_new(shared_ptr<const Data> & sp) {
    return sp.unique() ? const_ptr_cast<Data>(std::move(sp)) :
           sp          ? make_shared<Data>(*sp) :
           /* else */    make_shared<Data>() ;
  }
\end{lstlisting}

But this fails in multithreaded contexts, as \texttt{\sptr::unique()}
is only approximate. Besides, \texttt{\sptr::unique()} is deprecated
as of C++17.

The problem, of course, is that a new strong reference may be created
as a copy of \texttt{sp} in between the calls to \texttt{unique()} and
the move from \texttt{sp}, which will lead to a data race on
\texttt{sp->foo} further down the road, as the return value of
\texttt{clone\_or\_new()} is assumed, by \texttt{set\_foo()}, to
represent an exclusive owner.

We could try to define the problem away: assuming \texttt{sp.unique()}
returns \texttt{true}, we're looking at the sole remaining strong
reference. While the non-const function \texttt{set\_foo()} executes,
we can assume that no new copies of \texttt{sp} are taken, because the
only way to do so would be from another thread. We can declare such
use to be outside the contract of the function, along the lines of
``to call mutators, you need to externally synchronise'' as part of a
``const access is thread-safe'' policy.

And this view would be correct if it wasn't for \texttt{weak\_ptr},
which can upgrade to a strong reference at any time.

It follows that the condition for when we are able to re-use the
existing \sptr{} in \texttt{clone\_or\_new} is \emph{not}
\texttt{unique() == true}, but ``weak reference count is one'', iow: we
control the only \sptr{} and there are no \texttt{weak\_ptr}s registered
with it. But we have no API to check for this implementation detail of
\sptr.

Enter \sptrlockx.

This function returns a new owning pointer that represents the sole
owner of the data, setting \texttt{*this} to \texttt{nullptr} upon
success. The effects are atomic.

Our example function then becomes:

\begin{lstlisting}[language=C++,caption={\texttt{clone\_or\_new()} using \lockx}]
  static auto clone_or_new(shared_ptr<const Data> & sp) {
    auto uniq = const_pointer_cast<Data>(sp.lock_exclusive());
    if (!uniq)
      uniq = sp ? make_shared<Data>(*sp) : make_shared<Data>();
    return uniq;
  }
\end{lstlisting}

\subsection{Implementability}

All implementations the authors have have come across share the split
into a weak reference count, ref-counting the control block, and a
strong reference count, ref-counting the lifetime of the payload
object.

In all implementations, except \texttt{QSharedPointer}, the set of all
\sptr{} instances are counted as one (1) in the weak reference count.

The condition for when \lockx{} succeeds is thus \texttt{weak\_ref == 1
  \&\& strong\_ref == 1}.

\textit{Proof:} Since \texttt{strong\_ref == 1}, there is exactly one
\sptr{} (\texttt{*this}). Since \texttt{weak\_ref == 1}, there is
either one \wptr{} and no \sptr{} (contradicting the existence of
\texttt{*this}), or there is at least one \sptr{} and no
\wptr{}s. Taken together, it follows that there is exactly one \sptr{}
(\texttt{*this}) and no \wptr{}s.\hfill$\blacksquare$

A simple simultaneous relaxed atomic load of \texttt{weak\_ref} and
\texttt{strong\_ref} suffices to check for exclusive ownership, since
taking a new copy of \texttt{*this} while a non-const member function
is executing is already undefined behaviour, so the implementation of
\lockx{} can assume that when \texttt{weak\_ref} contains 1, it will
stay that way for the remainder of the function.

No stronger memory ordering than relaxed is needed, either, since we
don't touch the referenced data, in fact we change nothing except
swapping pointers from \texttt{*this} to the return value.

The advantage of this implementation is that only \lockx{} needs to be
added; all other operations, including \sptr/\wptr{} and \wptr/\sptr{}
conversions, can remain unchanged.

\emph{NB: We'd like to ask for guidance from implementers about the
  general feasibility of this. The implementation sketched above
  requires a double-word relaxed atomic load to simultaneously
  determine that \texttt{weak\_refs == 1} and \texttt{strong\_refs ==
    1}, to exclude a second thread permanently converting \sptr{}s
  into \wptr{}s and vice versa, but otherwise leaves the existing
  implementation of \sptr{} alone (ABI-compatible). On the other end
  of the spectrum, \texttt{weak\_refs} could count inclusive
  \texttt{strong\_refs} like \texttt{QSharedPointer} already does. In
  this case, \lockx{} is trivial\cite{qsp}, but both ABI and
  performance characteristics change. Maybe there is some
  middle-ground here, other protocols that can be used on platforms
  without double-word atomic operations, and don't require every
  \sptr{} copy to up two ref counts?}

\iffalse

\subsubsection{Exclusively-Counted Implementations}

This implementation is prone to a version of the ABA-problem: Assuming
we cannot check \texttt{weak\_ref} and \texttt{strong\_ref}
simultaneously, another thread could be continuously creating \wptr{}s
from \sptr{}s from \wptr{}s, \ldots, while \lockx{} tries to determine
whether it should succeed or fail.

Before looking at possible solutions, we first note that false
negatives (\lockx{} failing where it actually could succeed) is
perfectly ok, since the runtime needs to handle both outcomes, anyway.

However, spurious \texttt{\wptr::lock()} failures are not acceptable,
ie.\ calling \texttt{lock()} on an associated \wptr{} must not fail
(because a \sptr{} still exists). That implies that we cannot simply
emulate the \sptr{} destructor's operations on the reference counts.

The implementation could use a two-word atomic load, if the
architecture provides it:

\begin{lstlisting}[language=C++,caption={Possible implementation of \lockx{} in an exclusively-counted implementation, using DW relaxed load}]
  template <typename T>
  class shared_ptr {
    struct control_block {
      struct alignas(2*sizeof(long)) __refs {
        std::atomic<long> strong;
        std::atomic<long> weak;
      };
      std::atomic<__refs> refs;
      static_assert(std::atomic<__refs>::is_always_lock_free);
      ~~~
    };
    control_block *cb;
    T *v;
  public:
    shared_ptr lock_exclusive() noexcept {
      if (cb) {
          auto refs = cb->refs.load(std::memory_order_relaxed);
          if (refs.weak == 1 && refs.strong == 1) {
            // ie. one shared_ptr (*this) and zero weak_ptrs
            return std::move(*this);
          }
      }
      return {};
    }
  };
\end{lstlisting}

Note that we don't need double-word CAS, only a relaxed atomic load.
However, if that is not available except in a CAS operation, a DWCAS
could be used instead:

\begin{lstlisting}[language=C++,caption={Possible implementation of \lockx{} in an exclusively-counted implementation, using DWCAS}]
  template <typename T>
  class shared_ptr {
    ~~~same as before~~~
  public:
    shared_ptr lock_exclusive() noexcept {
      if (cb) {
          auto refs = control_block::_refs{1, 1};
          if (cb->refs.compare_and_swap_weak(refs, refs, std::memory_order_relaxed)) {
            // ie. one shared_ptr (*this) and zero weak_ptrs
            return std::move(*this);
          }
      }
      return {};
    }
  };
\end{lstlisting}

If no double-word atomic operations are available, this proposal
cannot be implemented without also changing other functions of
\sptr/\wptr. One option would be to half the size of the ref-counts,
and use a normal atomic load.

It would probably be a non-starter to add locking to implementations
that didn't formerly need it.

\emph{NB: We'd like to ask for guidance from implementers about the
  general feasibility of this. The alternative would be to make use of
  this function in combination with \wptr{} UB.}

This is equivalent to the inclusively-counted version, so the same
arguments apply for why relaxed memory ordering suffices and
\texttt{refs == \{1, 1\}} cannot legally change to something else
while \lockx{} executes.

Like in the inclusively-counted case, the advantage of this
implementation is that only \lockx{} needs to be added, and only
\lockx{} users pay the price of a double-word atomic load, if there is
any. All other operations, including \sptr/\wptr{} and \wptr/\sptr{}
conversions, can remain unchanged.

On architectures with no double-word atomic load, other operations may
need to change, too, to accomodate some other, more complex form of
protocol.

\emph{NB: The authors have not looked into an alternative protocol,
  yet, as we would like to get guidance about the implementability of
  this proposal from implementers first}.

\fi

\section{Impact on the Standard}

Minimal. We propose to add a new member function to \sptr{}. No other
part of the standard is affected.

\section{Proposed Wording}

\subsection{Changes to \cite{cpp2a}}

In section \isref{util.smartptr.shared}:

\begin{itemize}
\item at the end of paragraph (1), before the synopsis, continue the last sentence with
  \begin{quotation}
    [does not own a pointer], and to be an \emph{exclusive owner} if
    it shares ownership with no other \sptr{} and there are no
    \texttt{weak\_ptr} objects referring to it.
  \end{quotation}
\item at the end of \textit{modifiers} section, add the function
  \begin{quotation}
    \texttt{[[nodiscard]] shared\_ptr lock\_exclusive() noexcept;}
  \end{quotation}
\end{itemize}

In section \isref{util.smartptr.shared.mod}:

\begin{itemize}
\item add new paragraph at the end:
  \begin{quotation}
    \texttt{[[nodiscard]] shared\_ptr lock\_exclusive() noexcept;}
    \begin{itemize}
    \item\textit{Returns:} \texttt{std::move(*this)} if \texttt{*this}
      is an \textit{exclusive owner (insert reference to
        \isref{util.smartptr.shared}/1)}, otherwise returns
      \texttt{\{\}}. This function executes atomically.
    \item\textit{Synchronization:} none.
    \end{itemize}
  \end{quotation}
\end{itemize}

\subsection{Feature Macro}

We propose to use a new macro,
\texttt{\_\_cpp\_lib\_shared\_ptr\_lock\_exclusive}, to indicate a
library's support for this feature.

\section{Design Decisions}

\subsection{Why not \texttt{is\_exclusive\_owner()}?}

One of the problems with \texttt{unique()} is that its result is
useless after the function has returned (and even before then). If
check (\texttt{is\_exclusive\_owner()} and action (move from the
object) are not one function, it opens a window where the check can
become false.

That said, we're open to explore the idea of having such a query
\emph{in addition} to \lockx.

\subsection{Why \sptr?}

One legitimate question is: why put this functionality into \sptr{}
instead of inventing a new type (pair of types)? In particular, the
difficulties of dealing with \wptr, which is probably not used at all
with \sptr{}s that are used for copy-on-write, make inventing a new
pair of types to represent shared and unique ownership
attractive. However, there is the problem with names. What to call
such new types if \sptr{} and \uptr{} are already taken? Also, \sptr{}
is probably already in use for CoW systems, so adding the
functionality there, even if not enitrely trivial, probably gives the
biggest bang for the buck.

That said, we are open to explore the alternative with two different
types as well, should LEWG prefer that approach.

\subsection{Dealing with \wptr{}s}

If, for the use-cases in which \lockx{} is useful, we do not envision
use of \wptr{} at all, why not make calling \lockx{} in the presence
of associated \wptr{}s undefined behaviour?

This is an attractive option, too, since it would mean that
implementations would become easier. In this case, we would not need
\lockx{} at all, though, since we could just un-deprecate
\texttt{unique()} instead and use the implementation in
Listing~\ref{lst:clone-or-new-using-unique}.

\subsection{Return Type}

To represent unique ownership, we customarily use \uptr, so \lockx{}
could conceivably return a \uptr. But \sptr{}'s custom, type-erased
deleter and the \makeshared{} optimization of co-locating the control
block and the payload object in a single memory allocation make this
unrealistically complex: The \uptr{} returned would need to have some
form of type-erased deleter template argument, making it effectively a
different type from \uptr, and mostly layout-compatible with a
\sptr{}.

\subsection{Alternative Function Names}

We have chosen to call the function \lockx{} to piggy-back on the
existing \texttt{\wptr::lock()} function, which performs a similar (in
particular, atomic), but different task. The \texttt{exclusive} part
is intended to show that what is returned is an exclusive owner, if
any, of the payload data. The word "exclusive" has been chosen over
"unique" to avoid unwanted connotation with \uptr.

Alternatives considered include:

\begin{description}
\item[\texttt{unique()}] This would have been the first choice if it
  wasn't already taken to mean \texttt{strong\_ref == 1}. The name
  also has said unwanted connotation with \uptr, though.
\item[\texttt{lock\_unique()}] Not preferred for the unwanted
  connotation with \uptr{}. Would be a good choice if \lockx{}
  returned an actual \uptr{}, though.
\item[\texttt{release()}] For symmetry with \texttt{\uptr::release()};
  but we do not release the resource here, but merely
  conditionally move it into a new \sptr{}.
\end{description}

\subsection{Atomic Shared Pointers}

\texttt{std::atomic<shared\_ptr>}, as added for C++20, is merely
a container for \sptr{}s; they are not themselves
\sptr{}s. E.g. \texttt{std::atomic<weak\_ptr>} does not have a
\texttt{::lock()} member function. We therefore do not propose to add
\lockx{} to \texttt{std::atomic<shared\_ptr>}, even though it would
probably benefit some use-cases.

\iffalse
\section{Acknowledgements}
\fi

\section{References}
\renewcommand{\section}[2]{}%
\begin{thebibliography}{9}
\bibitem[S.Parent]{Seasoning}
  Sean Parent\newline
  \emph{C++ Seasoning}\newline
  in: \emph{Going Native 2013}\newline
  \url{https://channel9.msdn.com/Events/GoingNative/2013/Cpp-Seasoning}

\bibitem[Qt]{qt}
  \url{http://www.qt.io}

\bibitem[API]{property_based}
  \emph{Property-Based APIs}\\
  in: \emph{Qt Wiki: API Design Principles}\\
  \url{https://wiki.qt.io/API_Design_Principles#Property-Based_APIs}

\bibitem[QSP]{qsp}
  Marc Mutz\newline
  WIP: QSharedPointer::lock\_exclusive PoC\newline
  \url{https://codereview.qt-project.org/c/qt/qtbase/+/223451/1/src/corelib/tools/qsharedpointer_impl.h}
  
\bibitem[N4810]{cpp2a}
  Richard Smith (editor)\newline
  \emph{Working Draft: Standard for Programming Language C++}\newline
  \url{http://wg21.link/N4810}
\end{thebibliography}

\end{document}

